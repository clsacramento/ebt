\chapter{Problématique Réseau et SDN}
\label{chap-1}

%\section{Première section}\label{sec-1-1}

%\section{Deuxième section}\label{sec-1-2}

Ce chapitre va reprendre les problèmes réseaux rencontrés pour définir quels sont les besoins actuels dans le domaine. Dans ce chapitre je vais proposer une liste de requis pour une architecture réseau idéalement adapté aux applications actuels.
En connaissant les problèmes de l'architecture en place, la question que se pose est : si on repartait de zéro, comment on le ferrait ?

\section{Ossification de l'Internet}


\subsection*{Brain storming *}
Difficile d'innover, parce que on ne veut pas tester sur le trafique de production; on a différents équipements distribué exécutant des tâches spécifiques = complixité; administration compliquée à cause de la manque d'une vue globale du réseau,  configuration de chaque équipement un à un et très attachée au vendeur et au logiciel qui tourne dans l'équipement;  
\par
Ça c'est vrai pour l'internet, mais pareil pour les réseaux LAN dès qu'ils deviennent un peu plus grands.
\par
Coûts opérationnels. 

\subsubsection{principles and priority *}
Complexity matters. The trouble with software is that it can do anything, no matter how complex, convoluted, fragile, incomprehensible, and ill-judged.
Software engineers understand the cost of such complexity. Because the networking community underestimates the cost of complexity, it pays no attention to one of the most important problems of the current Internet, which is that it is much too difficult to build, deploy, and maintain networked applications.


From the viewpoint of Internet users and application programmers, there are requirements that sometimes equal or exceed performance, availability, and efficiency in priority. These include ease of use, correctness, predictability, and modularity. The use of functional modeling and formal reasoning to help meet such requirements is all-but-unknown in the networking community.


\subsection*{Fin brain storming *}
A vouloir autoriser et même encourager les utilisateurs à innover sur son architecture, l'internet s'est fait dépasser par son propre succès. La croissance explosive des utilisateurs, du trafic, des applications et des menaces a apporté toute une série de problèmes. L'internet a été créé à une époque où les choses étaient plus simples, avec peu d'intervenants en coopération. Dès qu'il est devenu partie intégrante en plusieurs aspects de la société, diverses réclamations et défaillances sont apparues dans différents aspects. En début des années 2000, la communauté de chercheurs en réseau était dominé par une conviction de qu'un travail n'est utile que si ses résultats peuvent être facilement adoptés dans l'architecture existante. En essayant de travailler avec cette contrainte, les concepteurs ont réalisé que l'architecture courante rend la résolution de certains problèmes impossible. \cite{InternetEvolutionRoleSoftwareEngineering}

%"The classic Internet architecture is a victim of its own success. Having succeeded so well at empowering users and encouraging innovation, it has been made obsolete by explosive growth in users, traffic, applications, and threats." The range of problems observed today is not surprising.

%The Internet was created in simpler times, among a small club of cooperating stakeholders. As the Internet becomes part of more and more aspects of society, it will inevitably be subject to more demands from more stakeholders, and be found deficient in more ways

%In the first half of the 2000s, the research climate was dominated by the belief that research is pointless unless its results can be adopted easily within the existing Internet. Attempting to work within this constraint, people realized that the current architecture makes solving some problems impossible.



\par
On se rend compte qu'il y q une haute barrière pour l'adoption de nouvelles idées dans le domaines des réseaux à cause de l'énorme base d'équipements et de protocoles installée ainsi qu'en raision d'une certaine résistence à les expérimenter en parrallele au trafic du mode de production. Finalement, il ne reste pratiquement pas de moyen concret pour tester de nouveaux protocoles réseau dans une configuration assez réaliste pour assurer et distribuer leurs déploiements. Par conséquant, la majorité des nouvelles idées émises dans le cadre de la recherche en "réseaux informatiques" finissent sans essais et sans tests, ce qui confirme la croyance répandue que l'infrastructure réseau "s'est ossifiée". \cite{OpenFlowStanford} 




\section{Configuration d'équipements distribués et spécifiques}

\subsection*{Brain storming *}

Même dans un réseau LAN de porte moyen,  on compte avec plusieurs équipements qui réalisent des fonctions spécifiques. Ces équipement doivent être configurable un à un donc pour atteindre à un objectif pour le réseau, tous la configuration de chaque équipement doit être orchestrée pour aboutir ce besoin. Il est difficile de mettre un place une configuration centralisée.

\par
Chaque équipement a été conçu pour réaliser une fonctionnalité spécifique. Pour toute correction de bug ou extension de ces fonctions,  il est nécessaire que le vendeur mette en place une mis à jour logiciel tenant en compte les modifications souhaités ou alors il faut acheter un nouveau équipement. 

\subsection*{Fin brain storming *}



\section{Clean-slate Internet}

Re-partir de zéro. En le faisant, quels sont les caractéristique de l'architecture ?


Cette problématique a amené scientistes et les ingénieurs impliqués à concevoir \gls{sdn}. \gls{sdn} est un nouveau \glslink{paradigme}{paradigme} réseau qu'on fait actuellement en cours de développer pour adapter l'infrastructure existante au nouveau scénario.