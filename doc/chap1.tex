\chapter{Problématique Réseau et SDN}
\label{chap-1}

%\section{Première section}\label{sec-1-1}

%\section{Deuxième section}\label{sec-1-2}

\section{Présentation de la Problématique}

Il y a une extrêmement haute barrière pour l'entrée de nouvelles idées dans le domaine des réseaux à cause de l'énorme base d'équipements et de protocoles installée ainsi qu'une résistance d'expérimenter avec le trafique de production. Finalement, il ne reste pratiquement pas de moyen pratique pour expérimenter des nouveaux protocoles réseau dans une configuration assez réaliste pour assurer et distribuer leurs déploiements. Comme résultat, la majorité des nouvelles idées de la recherche en réseau finissent sans essais et sans tests, ce qui apporte la croyance répandue que l'infrastructure réseau "s'est ossifiée". \cite{OpenFlowStanford} \par
Cette problématique a amené scientistes et tout le personnel impliqué à concevoir \gls{sdn}. \gls{sdn} est un nouveau \glslink{paradigme}{paradigme} réseau qu'on fait actuellement l'effort développer pour adapter l'infrastructure existante au nouveau scénario.