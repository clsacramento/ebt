\chapter{Problématique Réseau et SDN}
\label{chap-1}

%\section{Première section}\label{sec-1-1}

%\section{Deuxième section}\label{sec-1-2}

\section{Ossification de l'Internet}


Ayant habilité et même encouragé les utilisateurs a innover sur son architecture, l'internet s'est faite obsolète par son propre succès. La croissance explosive des utilisateurs, trafique, applications et menaces a apporté tout une gamme de problèmes. L'internet a été créée dans une époque plus simple, parmi peu d'intervenant en coopération. Dès l'internet devient partie de plus en plus aspects de la société, plusieurs demandes et déficiences apparaissent dans différents aspects. En début des années 2000, la communauté de chercheurs en réseau était dominé par une conviction de qu'un travail n'est utile que si ses résultats puissent être facilement adoptés dans l'architecture existante. En essayant de travailler avec cette contrainte, le personnel a réalisé que l'architecture courante rend la résolution de certains problèmes impossible. \cite{InternetEvolutionRoleSoftwareEngineering}

%"The classic Internet architecture is a victim of its own success. Having succeeded so well at empowering users and encouraging innovation, it has been made obsolete by explosive growth in users, traffic, applications, and threats." The range of problems observed today is not surprising.

%The Internet was created in simpler times, among a small club of cooperating stakeholders. As the Internet becomes part of more and more aspects of society, it will inevitably be subject to more demands from more stakeholders, and be found deficient in more ways

%In the first half of the 2000s, the research climate was dominated by the belief that research is pointless unless its results can be adopted easily within the existing Internet. Attempting to work within this constraint, people realized that the current architecture makes solving some problems impossible.



\par
On aperçoit cette haute barrière pour l'entrée de nouvelles idées dans le domaine des réseaux à cause de l'énorme base d'équipements et de protocoles installée ainsi qu'une résistance d'expérimenter avec le trafique de production. Finalement, il ne reste pratiquement pas de moyen pratique pour expérimenter des nouveaux protocoles réseau dans une configuration assez réaliste pour assurer et distribuer leurs déploiements. Comme résultat, la majorité des nouvelles idées de la recherche en réseau finissent sans essais et sans tests, ce qui apporte la croyance répandue que l'infrastructure réseau "s'est ossifiée". \cite{OpenFlowStanford} 


principles and priority
Complexity matters. The trouble with software is that it can do anything,
no matter how complex, convoluted, fragile, incomprehensible, and ill-judged.
Software engineers understand the cost of such complexity. Because the net-
working community underestimates the cost of complexity, it pays no attention
to one of the most important problems of the current Internet, which is that it
is much too difficult to build, deploy, and maintain networked applications.


From the viewpoint of Internet users and application programmers, there
are requirements that sometimes equal or exceed performance, availability, and
efficiency in priority. These include ease of use, correctness, predictability, and
modularity. The use of functional modeling and formal reasoning to help meet
such requirements is all-but-unknown in the networking community.







clean-slate internet




Cette problématique a amené scientistes et tout le personnel impliqué à concevoir \gls{sdn}. \gls{sdn} est un nouveau \glslink{paradigme}{paradigme} réseau qu'on fait actuellement l'effort développer pour adapter l'infrastructure existante au nouveau scénario.