\documentclass[a4paper,12pt,bibliography=totoc,index=totoc,twoside,francais]{scrbook}
%\documentclass[a4paper,12pt,DIVcalc,bibliography=totoc,index=totoc,twoside,francais]{scrbook}
\KOMAoptions{titlepage,chapterprefix,open=right}
%\KOMAoptions{bibliography=totoc,index=totoc}
%\addtokomafont{chapter}{\rmfamily}
%\addtokomafont{section}{\rmfamily}

\usepackage[utf8]{inputenc}
\usepackage[T1]{fontenc}
\usepackage{lmodern}
\usepackage{graphicx}
\usepackage[automark,headsepline]{scrpage2}
\usepackage[style=numeric,sorting=none,backend=biber]{biblatex}
\usepackage{csquotes}
\usepackage{xspace}
\usepackage[autolanguage]{numprint}
\usepackage{array}
\usepackage{booktabs}
\usepackage[table,svgnames,dvipsnames]{xcolor}
\usepackage[final]{pdfpages} 
\clubpenalty=5000
\widowpenalty=5000

%\usepackage{lipsum}
%\bibliography{biblio}

\usepackage[backend=biber]{biblatex}
\addbibresource{biblio.bib}

%\usepackage{hyperref} 
\usepackage[pdfauthor={Cynthia Lopes do Sacramento}, pdftitle={SDN : Software-Defined Networking}]{hyperref}

\usepackage{makeidx}
\makeindex


%\usepackage[xindy={language=french,codepage=utf8},acronym,toc=true,nonumberlist]{glossaries}
%\usepackage[acronym,toc=true,nonumberlist]{glossaries}
\usepackage[toc=true,acronym,nopostdot,nonumberlist]{glossaries}
%\usepackage[toc=true,acronym]{glossaries}

\makeglossaries


\let\Oldgls\gls%Transformation de la commande \gls en \Oldgls
\let\Oldglslink\glslink%Transformation de la commande \gls en \Oldgls
\let\Oldglspl\glspl%Transformation de la commande \gls en \Oldgls

% Création de la nouvelle commande \gls
\renewcommand{\gls}[1]{%
\textbf{\Oldgls{#1}}%
}
% Création de la nouvelle commande \gls
\renewcommand{\glslink}[2]{%
\textbf{\Oldglslink{#1}{#2}}%
}
% Création de la nouvelle commande \gls
\renewcommand{\glspl}[1]{%
\textbf{\Oldglspl{#1}}%
}

%\newacronym{rtfm}{RTFM}{Read the f\dots manual}

\newacronym{sdn}{SDN}{Software-Defined Networking : Réseau Informatique Défini par Logiciel}

\newacronym{ti}{TI}{Technologie de l'Information}

\newacronym{si}{SI}{Système d'Information}

\newacronym{nfv}{NFV}{Network Functions Virtualization, Virtualisation des fonctions réseau}

\newacronym{onf}{ONF}{Open Networking Foundation}

\newacronym{nos}{NOS}{Network Operating System, Système d'exploitation réseau}

\newacronym{vm}{VM}{Virtual Machine, Machine Virtuelle}

\newacronym{ip}{IP}{Internet Protocol, Protocole d'Internet}

\newacronym{vlan}{VLAN}{Virtual Local Area Network, Virtual LAN}

\newacronym{lan}{LAN}{Local Area Network, Réseau local}

\newacronym{wan}{WAN}{Wide Area Network, Réseau étendu}

\newacronym{nat}{NAT}{Network Address Translation, Traduction d'adresse réseau}

\newacronym{dhcp}{DHCP}{Dynamic Host Control Protocol, Protocole pour la configuration automatique d'hôte}

\newacronym{dns}{DNS}{Domain Name System, Système de noms de domaine}

\newacronym{mpls}{MPLS}{MultiProtocol Label Switching, Commutation multi-protocoles par étiquettes}

\newacronym{ids}{IDS}{Intrusion Detection System : Système de Détection d'Intrusion}

\newacronym{ips}{IPS}{Intrusion Prevention System : Système de Prévention d'Intrusion}

\newacronym{one}{ONE}{Open Network Environment, Environnement Réseau Ouvert}

\newacronym{api}{API}{Application Programming Interface, Interface de Programmation}

\newacronym{asic}{ASIC}{Application Specific Integrated Circuit, Circuit intégré pour application spécifique}

\newacronym{iaas}{IaaS}{Infrastructure as a Service, Infrastructure en tant que service}

\newacronym{http}{HTTP}{HyperText Transfer Protcol : Protocole de transfert de hypertexte}

\newacronym{qos}{QoS}{Quality of Service, Qualité de service}

\newacronym{aci}{ACI}{Application Centric Infrastructure, Infrastructure centrée sur les applications}


\newglossaryentry{paradigme}
{
  name=Paradigme,
  text=paradigme,
  description={Un paradigme dénote une collection de règles, standards et exemples de pratiques scientifiques, partagés par un groupe de scientifiques. 
  Sa genèse et continuation de la tradition de recherche sont conditionnées à l'engagement et au consensus qui en découle. \cite{paradigmdef}
  D'après Dosi \cite{newparadigm}, quand un nouveau paradigme technologique apparaît, il représente une discontinuité ou un changement de la manière de penser. Ce changement apporté par le paradigme est souvent lié à une sorte d'innovation radicale qui applique une nouvelle technologie. Dans ce document, le terme paradigme sera employé dans ce sens d'innovation et application de nouvelle technologie.   }
}

\newglossaryentry{scalability}
{
  name=Scalabilité,
  text=scalabilité,
  description={ Terme provenant de l'anglicisme \textit{scalability}qui exprime la capacité d'être mis à échelle. En informatique cela désigne la capacité d'un système, d'un réseau ou un processus de gérer l'augmentation ou la réduction de la charge de manière à pouvoir l'accommoder. \cite{scalability}. Le terme est souvent employé tant que extensibilité, évolutivité ou passage à l'échelle, mais il n'y <<a pas d'équivalent communément admis en français >>. \cite{chevance2001serveurs}  }
}

%n'a pas d'équivalent communément admis en français


\newglossaryentry{abstraction}
{
  name=Abstraction,
  text=abstraction,
  description={  Dans l'informatique, l'abstraction est un terme souvent employé pour désigner le mécanisme et la pratique qui réduit et factorise les détails négligeables de l'idée exprimée pour que l'on puisse focaliser sur moins de concepts à la fois.
     On voit aussi l'idée de couches d'abstraction utilisée comme un moyen pour gérer la complexité des systèmes informatiques où les couches correspondent à des niveaux de détails appliqués. \cite{AbstractionCS}
  }
}


%Scalability is the ability of a system, network, or process to handle a growing amount of work in a capable manner or its ability to be enlarged to accommodate that growth.

\newglossaryentry{bigdata}
{
  name=Big Data,
  description={Big Data est un terme appliqué aux ensembles de données dont la taille est au-delà de la capacité des outils logiciels communs de les capturer, les gérer et les traiter. Une nouvelle classe de technologies et outils ont été développés pour surmonter le challenge de créer valeur commercial de la complexe analyse de ces données. Le terme est employé pour référer ce type de données ainsi que les technologies utilisées pour les stocker et les traiter.
  \cite{IMBigData}    }
}


\newglossaryentry{cluster}
{
  name=Cluster,
  text=cluster,
  description={  Cluster est un ensemble d'ordinateurs. }
}


\newglossaryentry{cloudcomputing}
{
  name=Cloud Computing,
  text=cloud computing,
  description={  Cloud Computing, ou informatique dans les nuages, est une évolution de la livraison de services \gls{ti} qui offre un moyen d'optimiser l'usage et le déploiement rapide de ressources. Cela se fait à travers les systèmes et solutions plus efficaces et \glslink{scalability}{scalables}, fournissant en niveau plus haut d'automatisation. Divers entreprises ont adopté le cloud computing et réalisent des signifiants avantages en agilité, réduction de coûts et support à la croissance du business. \cite{CloudComputingIntelVisionSpeeding}     }
}



\newglossaryentry{virtualisation}
{
  name=Virtualisation,
  text=virtualisation,
  description={  Pour diverses entreprise, l'infrastructure serveur virtualisée est la base sur laquelle le \glslink{cloudcomputing}{cloud} est construit. Initialement, les technologies de virtualisation ont permis aux data centers de consolider leurs infrastructures pour réduire les coûts. Avec le temps, l'intégration des technologies pour le management flexible de ressources a habilité leur allocation plus dynamique. Cela a aidé à réduire les coûts et a également augmenté la flexibilité et la performance. \cite{CloudComputingIntelVisionSpeeding} }
}



\newglossaryentry{datacenter}
{
  name=Data  Center,
  text=data center,
  description={  Centre de traitement de données.  }
}
\newglossaryentry{middlebox}
{
  name=Middlebox,
  text=middlebox,
  description={  Boîtier intermédiaire. Un middlebox est un serveur gardant des états au milieu de la communication entre deux hôtes. Ils se différencient des hôtes qui représentent les 'endpoints' de la communication. Ils sont encore différents des routeurs qui ne gardent pas d'états concernant les instances de communications. \cite{InternetEvolutionRoleSoftwareEngineeringRealInternet}  },
  plural=middleboxes
}

\newglossaryentry{openflow}
{
  name=OpenFlow,
  description={  Le protocole OpenFlow focalise en standardiser 1) l'interface entre entre les applications et le contrôleur  et 2) entre le contrôleur et les éléments de commutation. \cite{SurveySDNArchi} \cite{OpenFlowStanfordSwitch}  }
}


\newglossaryentry{controlplane}
{
  name=Plan de Contrôle,
  text=plan de contrôle,
  description={  Intelligence du réseau, ensemble des données locales utilisées pour établir les entrées des tableaux de commutation, qui sont utilisés par le plan de données pour effectuer la transmission du trafic entre les ports d'entrée et de sortie du dispositif. \cite{sdnbookControlDataPlanes} },
  plural={Plans de contrôles}
}

\newglossaryentry{dataplane}
{
  name=Plan de Données,
  text=plan de données,
  description={  Le plan de donnés traite les data-grammes entrants dans le média à travers une série d'opérations au niveau des liens qui collectent ces data-gramme et réalisent divers tests de cohérence basiques. Ensuite les data-grammes sont transférés en accord avec des tableaux pré-remplis par le \gls{controlplane}.  \cite{sdnbookControlDataPlanes}},
  plural={plans de données}
}

\newglossaryentry{opensource}
{
  name=Open Source,
  text=open source,
  description={ Logiciel avec code source ouvert, qui peut donc être utilisé librement, modifié et partagé par quelqu'un. Un logiciel open source est développé par plusieurs personnes et distribué sous des licences qui se conforment à la définition d'open source.  \cite{OpenSource}  }
}

\newglossaryentry{opendaylight}
{
  name=Open Daylight,
  description={ Association provoquée par Linux Foundation pour l'union des géants du marché réseau dans le but de développer un contrôleur SDN open source, pour l'innover, l'encourager et pour permettre son adoption accélérée. \cite{OpenDaylight} }
}


\newglossaryentry{fabric}
{
  name=Fabric,
  text=fabric,
  description={ L'idée du design réseau autour d'un \textit{fabric} est bien entendu par la communauté. Sous ce modèle, un \textit{fabric} est un système composant qui représente 'grossièrement (roughly)' la commutation. En théorie, un \textit{fabric} devrait être capable de supporter un nombre de conceptions de pointe y compris des schémas d'adressage et des modèles de politique. \cite{fabricExtending}}
}

\usepackage[english,francais]{babel}
\frenchbsetup{og=«, fg=»}



\pagestyle{scrheadings}

\begin{document}
\includepdf[pages={1-2}]{couverture-ebt.pdf}

\frontmatter
\begin{flushright}
It is a sad age when it is more difficult to break a prejudice than an atom.\\
Albert \bsc{Einstein}\\
\end{flushright}

\tableofcontents
\listoftables
\listoffigures

\mainmatter
%\addchap{Introduction}
\addchap{Introduction}

%\section{To Do}
%Parler de la croissance (de l'explosion même) de l'utilisation de l'internet. Du hypertexte aux applications dynamiques. Puis la virtualisation et le cloud computing. Ensuite le big data. Montrer que ça évolue en grande vitesse.
%\par Expliquer que l'architecture et l'infrastructure réseau n'avaient pas été conçues pour ce scénario. Donc ça commence à se saturer ne correspondant plus aux besoins actuels. Les choses sont plus éphémères, il y a besoin de quelque chose plus flexible, adaptable. Et l'architecture actuelle pose des difficultés pour l'expérimentation des nouveaux protocoles, services, applications etc. 
%\par Introduire SDN comme une réponse à cette problématique. Expliquer brièvement ce que c'est et pour quoi cela apporte une évolution. Montrer comment ça pourrait être utilisé pour répondre aux besoins actuels.
%\par Expliquer les objectifs du texte : ré-définir SDN, présenter les enjeux et les cas d'utilisation et un état de l'art des technologies qui sont sorties.
%\par Présenter la méthodologie, le développement du texte et de quoi va parler chaque section. 
 

%\section{Évolution de l'utilisation Internet pendant une décade}

L'internet a évolué de trois manières importantes dans les dix dernières années. Le contenu a évolué de texte et pages web relativement statiques à un contenu multimédia haut-débit exigeant une latence réduite. L'utilisation s'est rapidement mondialisée, par exemple le débit international servant l'Afrique a augmenté de 1.21Gbit/s en 2001 à 570.92Gbit/s en 2011 \cite{InternetGlobalGrowth}. L'accès a étendu des ordinateurs de bureau à une variété de nouveaux dispositifs, par exemple le trafique global des donnés mobiles a augmenté de 70\% en 2012 \cite{CiscoVNI2013}. Fait important, la rapidité de telle évolution technologique et sa adoption est sans précédent dans l'histoire de l'humanité. 


\par
La capacité d'évolution pour s'adapter aux nouvelles exigences des usagers est recourant dans l'histoire de l'Internet. \cite{InternetSustainGrowth}
%La croissance accélérée de l'accès de partout, notamment dans les pays en développement et la rapide augmentation de l'utilisation par les utilisateurs existants, dirigées par les contenus multimédias et application machine-à-machine sont des 
Par contre, dans le scénario actuel on aperçoit la croissance accélérée de l'accès de partout, notamment dans les pays en développement et la rapide augmentation de l'utilisation par les utilisateurs existants, dirigées par les contenus multimédias et les applications machine-à-machine. Par exemple, en moins de deux ans depuis sa parution, plus de 50 millions de personnes on partagé plus d'un milliard de photos sur Instagram \cite{deuxAnsInstagram}.
Dans ce contexte, on met en cause sa capacité de continuer à fournir l'infrastructure nécessaire.
\par
Des nouvelles technologies et concepts émergent pour répondre aux nouveaux besoins de ces utilisateurs qui exigent de plus en plus haut-débit et réduction de latence. Le \gls{bigdata} a modifié le traitement des données pour permettre les entreprises de gérer la massive quantité de données manipulées. \cite{IMBigData} Le \gls{cloudcomputing} et la \gls{virtualisation} ont apporté une nouvelle approche pour le management et hebergement de ressources de \gls{ti} dans le but de les rendre plus agiles, plus efficients, plus sécurisés et plus flexibles tout en réduisant les coûts. \cite{CloudComputingIntelVision}. Pour accompagner ces évolutions, une innovation technologique dans le domaine des réseaux informatiques est requise. \cite{InternetEvolutionRoleSoftwareEngineering}
\par
Cette problématique a amené scientistes et tout le personnel impliqué à concevoir \gls{sdn}. \gls{sdn} est un nouveau \glslink{paradigme}{paradigme} réseau qu'on fait actuellement l'effort développer pour adapter l'infrastructure existante au nouveau scénario.\cite{OpenFlowStanford} Ce document a donc pour but d'exploiter cette solution et analyser les approches qui ont été faites dans ce domaine. Un état de l'art des technologies sorties pour déployer \gls{sdn} aussi bien que divers cas d'utilisation avec les enjeux seront présentés.
\par
[un paragraphe pour le plan du texte (de quoi parle chaque section)]
\par
[Un paragraphe pour conclure l'intro et laisser les pistes de mon point de vue et les conclusions trouvées].




\chapter{Problématique Réseau et SDN}
\label{chap-1}

%\section{Première section}\label{sec-1-1}

%\section{Deuxième section}\label{sec-1-2}

%Ce chapitre va reprendre les problèmes réseaux rencontrés pour définir les besoins actuels dans le domaine. Une liste de requis pour une architecture réseau idéalement adapté aux applications actuelles sera proposée.
%En connaissant les problèmes de l'architecture en place, la question que se pose est : si on repartait de zéro, comment on le ferrait ?

%\section{Pression pour l'expansion de l'architecture réseau }

\section{Ossification de l'internet face au besoin d'expansion}



%"The classic Internet architecture is a victim of its own success. Having succeeded so well at empowering users and encouraging innovation, it has been made obsolete by explosive growth in users, traffic, applications, and threats." The range of problems observed today is not surprising.

%The Internet was created in simpler times, among a small club of cooperating stakeholders. As the Internet becomes part of more and more aspects of society, it will inevitably be subject to more demands from more stakeholders, and be found deficient in more ways

%In the first half of the 2000s, the research climate was dominated by the belief that research is pointless unless its results can be adopted easily within the existing Internet. Attempting to work within this constraint, people realized that the current architecture makes solving some problems impossible.


À vouloir autoriser et même encourager les utilisateurs à innover sur son architecture, l'internet s'est fait dépasser par son propre succès. La croissance explosive des utilisateurs, du trafic et des applications a apporté toute une série de problèmes, 
%L'internet a été créé à une époque où les choses étaient plus simples, avec peu d'intervenants en coopération. Dès qu'il est devenu partie intégrante en plusieurs aspects de la société, diverses réclamations et défaillances sont apparues, 
tant que l'exhaustion des adresses IPv4 disponibles et les menaces aux réseaux locaux privés. 
\par
En réponse à ces questions, on a introduit dans l'architecture des \glspl{middlebox}, par exemple les NATs et les Firewalls, à un prix : la complexité. Le logiciel dans ces systèmes est capable d'atteindre n'importe quel objectif sous réserve de devenir excessivement complexe, fragile, incompréhensible et mal jugé. Parce que les coûts de la complexité ont été négligés lors de l'évolution de l'internet, les applications en réseau sortantes ont été rendues difficiles à concevoir, à mettre en place et à maintenir. \cite{InternetEvolutionRoleSoftwareEngineeringRealInternet}

Actuellement l'internet compte avec une énorme base d'équipements et de protocoles installée. Avec le système d'adressage IPv4 ce réseau peut interconnecter jusqu'à quatre milliards d'équipements. Capacité dont le limite est proche d'être atteint, confirmé par le développement du protocole IPv6 qui nous permettrait d'augmenter l'espace d'adressage par des centaines de milliards de fois. \cite{ICANNIPv6Important} 

L'internet est vu aujourd'hui comme une infrastructure critique de la société, tel que le transport ou l'électricité.  Cela provoque une résistance aux essais des nouvelles applications en parallèle à celles en mode de production. 
Ce fait a mené la communauté de chercheurs en réseau à se faire dominée par une conviction de qu'un travail n'est utile que si ses résultats peuvent être facilement adoptés dans l'architecture existante. En essayant de travailler avec cette contrainte, les concepteurs ont réalisé que l'architecture courante rend la résolution de certains problèmes impossible. \cite{OpenFlowStanfordOssification} \cite{SurveySDNIntro}

%illustrer les problèmes, donner un apperçu



%Complexity matters. The trouble with software is that it can do anything, no matter how complex, convoluted, fragile, incomprehensible, and ill-judged. Software engineers understand the cost of such complexity. Because the networking community underestimates the cost of complexity, it pays no attention to one of the most important problems of the current Internet, which is that it is much too difficult to build, deploy, and maintain networked applications.

%From the viewpoint of Internet users and application programmers, there are requirements that sometimes equal or exceed performance, availability, and efficiency in priority. These include ease of use, correctness, predictability, and modularity.

%The routing table in a typical router now has 300,000 entries, and these must be stored in the fastest, most expensive types of memory to maintain routing speed.

On se rend compte que l'adoption de nouvelles idées dans le domaine des réseaux reste complexe.  Finalement, les ingénieurs comptent avec peu de moyen concret pour tester de nouveaux protocoles réseau dans une configuration assez réaliste pour assurer et distribuer leurs déploiements. Par conséquent, la majorité des nouvelles idées émises dans le cadre de la recherche en "réseaux informatiques" finissent sans essais et sans tests. Cette barrière à l'évolution en face des besoins d'expansion des utilisateurs actuels confirme la croyance répandue que l'infrastructure réseau "est en phase d'ossification". \cite{OpenFlowStanfordOssification} 

%quantifier enorme.

\section{Management Réseau : Pénible et Complexe}

%\subsection*{Brain storming *}

%Même dans un réseau LAN de porte moyen,  on compte avec plusieurs équipements qui réalisent des fonctions spécifiques. Ces équipement doivent être configurable un à un donc pour atteindre à un objectif pour le réseau, tous la configuration de chaque équipement doit être orchestrée pour aboutir ce besoin. Il est difficile de mettre un place une configuration centralisée.

%\par Chaque équipement a été conçu pour réaliser une fonctionnalité spécifique. Pour toute correction de bug ou extension de ces fonctions,  il est nécessaire que le vendeur mette en place une mis à jour logiciel tenant en compte les modifications souhaités ou alors il faut acheter un nouveau équipement. 

%\par Difficulté de faire évoluer (lack of scalability or inability to scale) ; Complexité générant une résistance à l'innovation ; Dépendance du vendeur ; politiques inconsistantes

%Even with the help of autonomous and intelligent agents and network management software, the job of a network administrator is important and complicated. They must balance the different network management areas to make sure their system is properly configured and maintained. 

%The Internet was not designed with management in mind, yet the administrators of today’s networks face critical problems of configuration, traffic engineering, routing policy, and failure diagnosis. Their tools for understanding network traf- fic are poor, and their mechanisms for controlling network operations do not offer a predictable relationship between cause and effect.

%Internet routing is beginning to have serious problems of scale. The routing table in a typical router now has 300,000 entries, and these must be stored in the fastest, most expensive types of memory to maintain routing speed.2 There are efforts to move toward a scheme in which a typical routing table has one entry per autonomous system, which points to a router that can route to all the addresses for which that autonomous system is responsible.

%Network configuration and installation requires highly-skilled personnel adept at configuration of many network elements. Where interactions between network nodes (e.g. switches, routers, etc.) are complex, a more systems-based approach encompassing elements of simulation is required. With the current programming interfaces on much of today’s networking equipment, this is difficult to achieve. In addition, operational costs involved in provisioning and managing large, multi-vendor networks covering multiple technologies have been increasing over recent years, whilst the pre- dominant trend in revenue for operations has been decreasing.

%\subsection*{Fin brain storming *}

Cette résistance empêchant l'innovation est contradictoire à la rapide croissance de l'internet qui impose une expansion de l'architecture réseau. 
Le besoin pour des services réseau et pour haut-débit augmente à un taux plus rapide que la disponibilité ou les revenues.



\begin{figure}[!h] %on ouvre l'environnement figure
\includegraphics[width=15cm]{images/IncreasingPressureOnNetworkInfra.png} %ou image.png, .jpeg etc.
\caption{ Pression croissante sur l'infrastructure réseau \cite{IBMManagingGrowingPainsNeed}} %la légende
\label{image_soleil} %l'étiquette pour faire référence à cette image
\end{figure} %on ferme l'environnement figure


% Figure 1: The need for network capabilities and bandwidth is expanding at a faster rate than either bandwidth availability or revenue to grow

\clearpage

Le plus une entreprise dépend d'un nombre croissant de dispositifs et gros volumes de données, les plus importante est la demande pour débit et expansion de l'infrastructure. La complexité de cette expansion des réseaux augmente la probabilité des interruptions de service dues à une faille humaine ou autre problème. Ce fait met en évidence l'importance de la disponibilité, fiabilité, performance et sécurité. L'efficacité et la réduction des coûts deviennent cruciales pour aboutir la mis en échelle de ces besoins, et donc le management assume en rôle primordial dans ce contexte. \cite{IBMManagingGrowingPainsNeed}
 

Malgré l'assistance des agents autonomes et intelligents ainsi que des logiciels pour le management réseau, la mission de l'administrateur réseau reste importante et compliquée. Il doit équilibrer les différentes tâches du management pour assurer que le \gls{si} soit proprement configuré et maintenu. \cite{CentralIssuesNetworkManagementConclusion}

Internet n'a pas été conçu pour le management, pourtant les administrateurs des réseaux d'aujourd'hui rencontrent des problèmes critiques de configuration, ingénierie du trafic, politique de routage et diagnostic de failles. Leurs outils pour analyse du trafic sont faibles et leurs mécanismes pour contrôler les opérations réseau ne facilitent pas la prédictibilité entre des relations cause-effet.

Le routage sur internet a un fort problème d'évolution. La table de routage d'un routeur typique en ce moment compte avec 300000 entrées qui doivent être stockés dans les plus rapides et plus couteux types de mémoire pour assurer la rapidité du routage.


%A new network model is required to support this.
\section{Un nouveau modèle réseau pour supporter ces évolutions}
%\section{Clean-slate Internet}

Re-partir de zéro. En le faisant, quels sont les caractéristique de l'architecture ?


Cette problématique a amené scientistes et les ingénieurs impliqués à concevoir \gls{sdn}. \gls{sdn} est un nouveau \glslink{paradigme}{paradigme} réseau qu'on fait actuellement en cours de développer pour adapter l'infrastructure existante au nouveau scénario.


%\section{Les requis d'un réseau idéalement adapté aux besoins courants}
\chapter{Réseaux programmables avec SDN}

Le but de ce chapitre est de (re)définir SDN et de présenter en quoi SDN répond aux besoins explicités dans le chapitre 1.
Ce chapitre répond aux questions : Qu'est-ce que SDN ? Qu'est-ce que se la propose ? Où on en est par rapport à SDN ?

\section{Séparation de l'intelligence (contrôle) de la commutation}

blabla...
On cherche à concevoir une architecture plus adaptée aux enjeux de la communication de l'actualité discutés dans le chapitre 1. Cette problématique a amené scientistes et les ingénieurs impliqués à concevoir \gls{sdn}. \gls{sdn} est un nouveau \glslink{paradigme}{paradigme} réseau qu'on est actuellement en cours de développer pour adapter l'infrastructure existante au nouveau scénario.


%SDN is described in this article with the Open Networking Foundation (ONF) [1] definition: “In the SDN architecture, the control and data planes are decoupled, network intelli- gence and state are logically centralized, and the underlying network infrastructure is abstracted from the applications.”

SDN est défini au long de l'article par  selon \gls{onf} \cite{SDNNewNormONFExecutiveSummary} : Dans l'architecture SDN, le plans de contrôle et de données sont découpés, l'intelligence et l'état du réseau sont logiquement centralisés, et l'infrastructure du réseau est donc abstraite des applications. 




\section{Tableaux de flux}

\section{Contrôleur}

\section{Un point sur la situation}
\subsection{ONF}

\subsection{OpenFlow - Protocoles standardisés}

%\chapter{Enjeux de SDN}

Ce chapitre va présenter quels sont les enjeux pour déployer SDN. Quels sont les problèmes que cette architecture peut poser et les propositions pour les surmonter.

\section{Contrôle centralisé vs distribué}

\section{Niveau de granularité}

\section{Politiques réactives vs pro-actives}

\section{Fonctions de Virtualisation du Réseau}
%\gls{nfv}
\chapter{Solutions SDN disponibles}

Le but de ce chapitre n'est pas de détailler chaque solution SDN émergeant, mais d'analyser les offres des principales constructeur du marché et leur positionnement pour les tendances qu'on peut espérer de SDN prochainement.

\section{Solution logiciel open source OpenDaylight}

\section{Écosystème SDN HP, solution virtualisée}

SDN Dev Center.
SDN App Store.

HP SDN Developer Kit


\section{Cisco ONE, hardware différentiel}

Cisco Open Network Environment (ONE) is a comprehensive solution to help networks become more open, programmable, and application-aware. The broad capabilities of Cisco ONE help meet the needs of numerous market segments, including emerging concepts such as software-defined networking (SDN).

\section{Brocade Ethernet Fabric, Fibre Channel support over Ethernet}
Virtual Cluster Switching
Pour les architectes du réseau et serveurs de data centre.

\section{VMWare NSX, from physical to logical services}

\section{Juniper MetaFabric Architecture}

\section{Citrix NetScaler, plate-forme ouverte dirigé par app }




\chapter{Solutions SDN disponibles}

Le but de ce chapitre n'est pas de détailler chaque solution SDN émergeant, mais d'analyser les offres des principales constructeurs du marché et leur positionnement pour les tendances qu'on peut espérer de SDN prochainement.
Mais d'abord il présente un point sur la situation de SDN, en précisant les standards et protocoles développés par des organisations supportant SDN.
https://www.opennetworking.org/sdn-resources/onf-products-listing

\section{Un point sur la situation}
%\subsection{ONF}

%The Open Networking Foundation (ONF) is a non‑profit, user‑driven organization dedicated to accelerating the adoption of open Software‑Defined Networking (SDN). We view SDN as a disruptive approach to networking that will change how virtually every company with a network operates.

%Launched in 2011 by Deutsche Telekom, Facebook, Google, Microsoft, Verizon, and Yahoo!, ONF is a nonprofit organization dedicated to rethinking networking, and quickly and collaboratively bringing to market SDN standards and solutions. ONF is accelerating the delivery and commercialization of SDN and fostering a vibrant market of products, services, applications, customers, and users. 


Open Networking Foudation (\gls{onf}) est une organisation non-profit et axée sur l'utilisateur dédiée à l'accélération de l'adoption ouverte de SDN. Cette organisation voit SDN comme une approche réseau qui va changer comment opère chaque entreprise avec un réseau.
\gls{onf} a été initiée en 2011 par Deutsche Telekom, Facebook, Google, Microsoft, Verizon et Yahoo! dans le but de repenser en collaboration les réseaux informatiques et rapidement apporter au marché les solutions et les standards SDN. Avec la collaboration de  grands experts mondiaux, ONF accélère la commercialisation de SDN en favorisant un vif marché de produits, services, applications, clients et utilisateurs. ONF compte aujourd'hui avec plus de 100 entreprises membres collaboratives de tout taille et variété. \cite{ONFOverview}

\gls{onf} a fait des efforts de standardiser le protocole \gls{openflow}. Ce protocole focalise en standardiser les interfaces entre les applications et le contrôleur et les interfaces entre le contrôleur et l'équipement de commutation. Grands noms de l'industrie (comme Cisco, Microsoft, Google etc.) ont réalisé des produits supportant OpenFlow, comme des switches. \cite{SurveySDNArchi}
%The Open Network Foundation (ONF) [3] has been trying to standardize the OpenFlow protocol. As the control plane abstracts network applications from underlying hardware infrastructure, they focus on standardizing the inter- faces between: (1) network applications and the controller (i.e. northbound interface) and (2) the controller and the switching infrastructure (i.e., southbound interface) which defines the OpenFlow protocol itself. 
%Many large companies (such as Cisco, Microsoft, Google, etc.) have produced OpenFlow-supported products (such as switches).

Le fort support de l'industrie, de la recherche et des académies que \gls{onf} et sa proposition de \gls{sdn}, \gls{openflow}, ont pu recueillir est assez expressif. Les résultats dans ces différents secteurs ont produit un nombre significatif de livrables dans la forme d'articles de recherche, d'implémentations de logiciels de référence et même de hardware. Il y a eu également des efforts de standardisation de SDN de la part d'autres organisations produisant des normes, comme IETF et IRTF. \cite{SurveySDNIntro}
%The strong support from industry, research, and academia that the Open Networking Foundation (ONF) and its SDN proposal, OpenFlow, has been able to gather is quite impres- sive. The resulting critical mass from these different sectors has produced a significant number of deliverables in the form of research papers, reference software implementations, and even hardware. So much so that some argue that OpenFlow’s SDN architecture is the current SDN de-facto standard. In line with this trend, the remainder of this section focuses on OpenFlow’s SDN model. 

%On the academic side, the OpenFlow Network Research Center [4] has been created with a focus on SDN research. There have also been standardization efforts on SDN at the IETF and IRTF and other standards producing organizations.



%\subsection{OpenFlow - Protocoles standardisés}
%\section{Dispositifs de commutation}
%The basic idea is simple: we exploit the fact that most modern Ethernet switches and routers contain flow-tables (typically built from TCAMs) that run at line-rate to im- plement firewalls, NAT, QoS, and to collect statistics. While each vendor’s flow-table is different, we’ve identified an in- teresting common set of functions that run in many switches and routers. OpenFlow exploits this common set of func- tions.




%\section{Contrôleur}
%Controllers. A controller adds and removes flow-entries from the Flow Table on behalf of experiments. For example, a static controller might be a simple application running on a PC to statically establish flows to interconnect a set of test computers for the duration of an experiment. In this case the flows resemble VLANs in current networks— providing a simple mechanism to isolate experimental traffic from the production network. Viewed this way, OpenFlow is a generalization of VLANs.



\section{Solution logiciel open source OpenDaylight}

\section{Écosystème SDN HP, solution virtualisée}

SDN Dev Center.
SDN App Store.

HP SDN Developer Kit


\section{Cisco ONE, hardware différentiel}

Cisco Open Network Environment (ONE) is a comprehensive solution to help networks become more open, programmable, and application-aware. The broad capabilities of Cisco ONE help meet the needs of numerous market segments, including emerging concepts such as software-defined networking (SDN).

\section{Brocade Ethernet Fabric, Fibre Channel support over Ethernet}
Virtual Cluster Switching
Pour les architectes du réseau et serveurs de data centre.

\section{VMWare NSX, from physical to logical services}

\section{Juniper MetaFabric Architecture}

%\section{Citrix NetScaler, plate-forme ouverte dirigée par app }


\addchap{Conclusion}

Même avec le succès incontestable de l'architecture d'internet, l'état de l'industrie réseau et l'essence de son infrastructure se retrouvent en phase critique. Il est généralement admis que les réseaux courants sont excessivement chers, compliqués à gérer, sujets aux blocages des fournisseurs et difficiles à faire évoluer. 

En effet, on constate un réel besoin de faire évoluer cette architecture mais, des résistances s'opposent à cette évolution dues à la complexité et la possible saturation du système. En réponse, les réseaux programmables ont été un objet intensif de recherche par la communauté. Les travaux dans ce domaine s'orientent vers la proposition de SDN, un nouveau paradigme transformant cette architecture.

L'approche SDN découpe le plan de contrôle et le plan données, offrant un contrôle et une vision centralisés du réseau. Cela peut apporter certains bénéfices comme le contrôle directement programmable, la simplification du hardware réseau et la simplification de l'ingénierie du trafic. En revanche, des défis d'implémentation sont à surmonter tels que la concentration des risques dans un contrôle physiquement centralisé, l'équilibre entre flexibilité et performance et les conditions d'interopérabilité.

La flexibilité apportée par SDN est telle que de nombreuses possibilités d'applications sont à imaginer. Essentiellement le management de data centers, le contrôle d'accès et de la mobilité pour les réseaux campus ainsi que  l'ingénierie du trafic pour les réseaux WAN.

Le marché suit de près les nouveautés dans le domaine et investit sur les technologies implémentant SDN. Les stratégies ne sont pas encore assez matures et les consommateurs potentiels attendent des offres plus consolidées. Cependant, des solutions innovantes commencent à surgir et certaines sociétés assument un rôle de tête dans le marché.

On s'aperçoit que l'ampleur des possibilités SDN même si elle présente un avantage en théorie, freine son adoption. En raison de la grande variété de concepts et produits, les consommateur hésitent toujours à prendre des décisions. En même temps, les grands fournisseurs cherchent à la fois à exploiter le nouveau marché et à protéger leurs solutions consolidées. Cette impasse même si elle est confirmée, ne semble pas être assez forte pour empêcher les échanges à long terme.

Au vu cette étude, il semblerait que dans un futur proche, les clients les plus informés et les plus prêt à innover vont commencer à déployer SDN. Vu que leurs expériences et résultats peuvent fortement impacter les choix des prochains consommateurs, il est possible que les premiers à se lancer seront ceux qui vont peindre le futur de la technologie des réseaux informatiques pour les prochaines années. Cette démarche peut représenter un risque au cas échéant, mais aussi l'opportunité d'en tirer des bénéfices plus durables et de prendre de parts plus larges du marché.

%\appendix
%\include{ann1}
%\include{ann2}
%\gls{sdn}
%\gls{paradigme}
%\gls{ti}

\backmatter
\nocite{*}
%\printbibliography

\printindex

\glsaddall

\printglossary[type=acronym,title=Acronymes,toctitle=Acronymes]
\printglossary[type=main,title=Glossaire,toctitle=Glossaire]
%\printglossaries

\cleardoublepage
\includepdf[pages={3-4}]{couverture-ebt.pdf}
\end{document}
