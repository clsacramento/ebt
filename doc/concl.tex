\addchap{Conclusion}

Même avec le succès incontestable de l'architecture d'internet, l'état de l'industrie réseau et l'essence de son infrastructure se retrouvent en phase critique. Il est généralement admis que les réseaux courants sont excessivement chers, compliqués à gérer, sujets aux blocages des fournisseurs et difficiles à faire évoluer. 

En effet, on constate un réel besoin de faire évoluer cette architecture mais aussi des résistances contraignant cette évolution dues à la complexité et la possible saturation du système. En réponse, les réseaux programmables ont été objet intensif de recherche par la communauté. Les travaux dans le domaine ont progressé vers la proposition de SDN, un nouveau paradigme transformant cette architecture.

L'approche SDN découpe le plan de contrôle et le plan données, offrant un contrôle et une vision centralisés du réseau. Cela peut apporter certains bénéfices comme le contrôle directement programmable, la simplification du hardware réseau et la simplification de l'ingénierie du trafic. En revanche, des challenges d'implémentation sont à surmonter tels que la concentration des risques dans un contrôle physiquement centralisé, l'équilibre entre flexibilité et performance et les conditions d'interopérabilité.

La flexibilité apportée par SDN est telle que de nombreuses possibilités d'applications sont à imaginer. Essentiellement le management de data centers, le contrôle d'accès et de la mobilité pour les réseaux campus ainsi que  l'ingénierie du trafic pour les réseaux WAN.

Le marché suit de près les nouveautés dans le domaine et investit sur les technologies implémentant SDN. Les stratégies sont pas encore assez matures et les potentiels consommateurs attendent des offres plus consolidées. Cependant, des solutions innovantes commencent à surgir et certaines organisations assument un rôle de tête dans le marché.

On s'aperçoit que l'ampleur des possibilités SDN même si un avantage en théorie, frein son adoption. Dû à la massive variété des concepts et produits, les consommateur hésitent toujours à prendre des décisions. Au même temps, les grands fournisseurs cherchent à la fois à exploiter le nouveau marché et à protéger leurs solutions consolidées. Cet impasse même si confirmé, ne semble pas être assez fort pour empêcher les échanges à long terme.

En caractère personnel, cette étude conclue qu'au futur proche, les clients les plus informés et les plus prêt à innover vont commencer à déployer SDN. Vu que leurs expériences et résultats peuvent fortement impacter les choix des prochains consommateurs, il est possible que les premiers à se lancer seront ceux qui vont peindre le futur de la technologie des réseaux informatiques pour les prochaines années. Cette démarche peut représenter un risque au cas échéant, mais aussi l'opportunité d'en tirer des bénéfices plus durables et de prendre de parts plus larges du marché.