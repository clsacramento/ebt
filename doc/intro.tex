\addchap{Introduction}

%\section{To Do}
%Parler de la croissance (de l'explosion même) de l'utilisation de l'internet. Du hypertexte aux applications dynamiques. Puis la virtualisation et le cloud computing. Ensuite le big data. Montrer que ça évolue en grande vitesse.
%\par Expliquer que l'architecture et l'infrastructure réseau n'avaient pas été conçues pour ce scénario. Donc ça commence à se saturer ne correspondant plus aux besoins actuels. Les choses sont plus éphémères, il y a besoin de quelque chose plus flexible, adaptable. Et l'architecture actuelle pose des difficultés pour l'expérimentation des nouveaux protocoles, services, applications etc. 
%\par Introduire SDN comme une réponse à cette problématique. Expliquer brièvement ce que c'est et pour quoi cela apporte une évolution. Montrer comment ça pourrait être utilisé pour répondre aux besoins actuels.
%\par Expliquer les objectifs du texte : ré-définir SDN, présenter les enjeux et les cas d'utilisation et un état de l'art des technologies qui sont sorties.
%\par Présenter la méthodologie, le développement du texte et de quoi va parler chaque section. 
 

%\section{Évolution de l'utilisation Internet pendant une décade}


On cherche à concevoir une architecture plus adaptée aux enjeux de la communication de l'actualité. Cette problématique a amené les scientifiques et les ingénieurs impliqués à concevoir \gls{sdn}. \gls{sdn} est un nouveau \glslink{paradigme}{paradigme} réseau qu'on est actuellement en cours de développer pour adapter l'infrastructure existante au nouveau scénario. Au long de l'article ce nouveau scénario sera discuté et SDN sera présenté en réponse à ses besoins.

L'internet a évolué de trois manières importantes dans les dix dernières années. 
\begin{itemize}
\item Le contenu a évolué de texte et pages web relativement statiques, il a progressé vers un contenu multimédia haut-débit exigeant une latence réduite. 
\item L'utilisation s'est rapidement mondialisée; par exemple le débit international servant l'Afrique a augmenté de 1.21Gbit/s en 2001 à 570.92Gbit/s en 2011 \cite{InternetGlobalGrowthImpactDevelopingCountries}.
\item  L'accès a étendu des ordinateurs de bureau à une variété de nouveaux dispositifs, comme pour les téléphones mobiles dont le trafic global des donnés a augmenté de 70\% en 2012. \cite{CiscoVNI2013}. 
\end{itemize}
Fait important: la rapidité d'une telle évolution technologique et son adoption est sans précédent dans l'histoire de l'humanité. 


\par
La capacité d'évolution pour s'adapter aux nouvelles exigences des usagers est recourrente dans l'histoire de l'Internet. 
%La croissance accélérée de l'accès de partout, notamment dans les pays en développement et la rapide augmentation de l'utilisation par les utilisateurs existants, dirigées par les contenus multimédias et application machine-à-machine sont des 
En revanche, dans le scénario actuel on voit poindre une croissance accélérée de l'accès de partout, notamment dans les pays en développement, ainsi qu'une rapide augmentation de l'utilisation en général, engendrée par les contenus multimédias et les applications machine-à-machine. Par exemple, en moins de deux ans depuis la parution d'Instagram, plus de 50 millions de personnes on partagé plus d'un milliard de photos desssus. \cite{deuxAnsInstagram}.
Dans ce contexte, on met en cause la capacité d'internet, comme il est, de continuer à fournir l'infrastructure nécessaire. \cite{InternetSustainGrowthIntro}
\par
De nouvelles technologies et concepts émergent pour répondre aux nouveaux besoins de ces utilisateurs qui exigent de plus en plus haut-débit et une latence réduite. Le \gls{bigdata} a modifié le traitement des données pour permettre les entreprises de gérer la quantité massive de données manipulées. \cite{IMBigData} Le \gls{cloudcomputing} et la \gls{virtualisation} ont apporté une nouvelle approche pour le management et l'hébergement de ressources de \gls{ti} dans le but de les rendre plus agiles, plus efficaces, plus sécurisés et plus flexibles tout en réduisant les coûts. \cite{CloudComputingIntelVision}. Pour accompagner ces évolutions, une innovation technologique dans le domaine des réseaux informatiques est requise. \cite{InternetEvolutionRoleSoftwareEngineeringConclusion}
\par
Cette problématique a amené scientifiques et tout les ingénieurs impliqués dans ce secteur à la conception de \gls{sdn}. \gls{sdn} est un nouveau \glslink{paradigme}{paradigme} réseau qui est actuellement développé en collaboration pour adapter l'infrastructure existante au nouveau scénario.\cite{OpenFlowStanford} Le présent document a donc pour but d'explorer cette solution et analyser les approches qui ont été faites dans ce domaine. Il propose un état de l'art des technologies parues pour déployer \gls{sdn} ainsi que divers cas d'utilisation dont les enjeux seront présentés.
\par
[un paragraphe pour le plan du texte (de quoi parle chaque section)]
\par
[Un paragraphe pour conclure l'intro et laisser les pistes de mon point de vue et les conclusions trouvées].


