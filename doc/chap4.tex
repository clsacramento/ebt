
\chapter{Des Applications de SDN et leurs possibilités}

%Ce chapitre a pour but de présenter ce qui apporte SDN, quelles sont les applications pratiques de ce nouvel paradigme. Ce ne sera pas exhaustive, mais c'est pour exemplifier. Cela permettra aussi d'avoir une idée de l'exploitation future de SDN.

Une infinité d'applications et de cas d'utilisation sont imaginables. Ce chapitre propose d'analyser les enjeux qu'on espère de répondre avec SDN et de présenter des applications plutôt ciblées sur ces expectatives. De cette analyse, les cas d'utilisation plus cohérents seront identifiés et ensuite détaillés.

 
%\section{Identifications des applications espérées par les principaux utilisateurs}

%In August and September of 2013 a survey was given to the subscribers of Webtorials.
Entre les mois d'août et septembre de 2013, Webtorials a réalisé un sondage auprès de ces abonnés. Une des enquêtes a été sur les challenges et opportunités qu'ils pensent pouvoir adresser avec SDN au sein des organisations de \gls{ti} typiques. Le tableau ci-dessous affiche la pourcentage de réponses positives pour chaque challenge. \cite{2013GuideSDNNVUseCases}

\begin{table}[!h]
\centering
\begin{tabular}{|p{12cm}|c|}
\hline 
\bf Challenge ou Opportunité & \bf Pourcentage \\ 
\hline 
Meilleure utilisation des ressources réseau & 51\% \\ 
\hline 
Simplifier la configuration de la QoS et de la sécurité  & 47\%  \\
\hline 
Réaliser l'ingénierie du trafic avec vision point-à-point du réseau & 44\% \\ 
\hline 
Évolution plus facile des fonctions réseau & 39\% \\ 
\hline 
Support dynamique au management de ressources virtuelles  & 38\% \\ 
\hline 
Établissement des réseaux Ethernet virtuels sans les limitations du fardeau de configuration des VLANs & 35\% \\ 
\hline 
Réduction de la complexité & 34\% \\ 
\hline 
Permettre les demandes dynamiques de services au réseau & 32\% \\ 
\hline 
Réduction les dépenses d'exploitation & 30\% \\ 
\hline 
Faire évoluer les fonctionnalités réseau plus rapidement s'appuyant sur les cycles de vie du développement logiciel & 27\% \\ 
\hline 
Implémentation plus facile de la QoS & 27\% \\ 
\hline 
Implémentation des fonctionnalités de sécurité plus effectives & 26\% \\ 
\hline 
Réduction des dépenses d'investissement de capital & 25\% \\ 
\hline 
Absence de challenges/opportunités pouvant être adressés par SDN & 3\% \\ 
\hline 
\end{tabular}
\caption{Opportunités et Challenges pouvant être adressés par SDN \cite{2013GuideSDNNVTable11}}
\end{table} 

\clearpage


\section{Management du Réseau et Contrôle d'Accès}
Association des flux aux groupe d'utilisateur permettant de définir les politiques d'accès à différents services.

\section{VLANs}
Réseaux isolés définis par flux.

\section{Clients mobiles sans fil VoIP}


