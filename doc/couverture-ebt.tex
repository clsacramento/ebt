\documentclass[a4paper,12pt,francais,twoside]{article}
%\documentclass[a4paper,12pt,twoside,francais,headinclude]{scrartcl}
%\documentclass[a4paper,12pt,twoside,francais,DIVcalc]{scrreprt,scrartcl}

\usepackage[utf8]{inputenc}
\usepackage[T1]{fontenc}
\usepackage{lmodern}
\usepackage{graphicx}
\usepackage{hyperref}
\usepackage{array}
\usepackage[margin=2cm]{geometry}
\usepackage[english,frenchb]{babel}
\parindent 0pt
\pagestyle{empty}

\newcommand{\promo}{RIE07 2012 -- 2014}
\newcommand{\datesoutenance}{10 avril 2014}
\begin{document}

\begin{titlepage}
%\raisebox{0.5cm}{\includegraphics[width=2.8cm]{logo-cnam}}\hfill%
%\raisebox{0.5cm}{\large Master Réseaux Informatiques d'Entreprise}\hfill%
%\includegraphics[width=3.0cm]{logo-fc}
\includegraphics[width=2.8cm]{logo-cnam}\hfill%
\large Master Réseaux Informatiques d'Entreprise\hfill%
\raisebox{-0.5cm}{\includegraphics[width=3.0cm]{logo-fc}}
\vspace{-0.5cm}
\begin{center}
\promo\\
\end{center}

\vspace{\stretch{.5}}
\begin{center} \large
Épreuve bibliographique tutorée\\
\end{center}
\begin{center}
présentée et soutenue le \datesoutenance{} par \\
\large Cynthia \bsc{Lopes do Sacramento}
\end{center}

\vspace{\stretch{1}}
\hrulefill
\begin{flushleft} \bfseries\sffamily\LARGE
  SDN : Software-Defined Networking
\end{flushleft}
\begin{flushright} \large
  Réseau Informatique Défini par Logiciel
\end{flushright}
\hrulefill


\vspace{\stretch{4}}
\begin{flushleft}
\begin{tabular}{ll}
Jury : & Romain \bsc{Kobylanski} \\
& François \bsc{Miller} \\
& Véronique \bsc{Panne} \\ \\
Tuteur : & Claude \bsc{Casery} \\
Entreprise : & Bull \\
\end{tabular}
\end{flushleft}
\end{titlepage}

%%%% quatrième de couverture
\cleardoublepage
\strut
\clearpage
Épreuve bibliographique tutorée \hfill Master Réseaux Informatiques d'Entreprise  -- 2014 \\

\begin{center} \large
SDN : Software-Defined Netowrking
\end{center}
\begin{flushright}
rédigé par Cynthia \textsc{Lopes do Sacramento}
\end{flushright}

\vspace{1cm}
\hrulefill
\begin{abstract}



De nouvelles technologies et concepts émergent pour répondre aux nouveaux besoins des utilisateurs qui exigent de plus en plus du haut-débit et une latence réduite, comme le Big Data pour le traitement des données le Cloud Computing pour le management et l'hébergement de ressources. 
Une évolution similaire est attendue dans le domaine des réseaux informatiques ce qui a mobilisé la communauté dans les projets de recherche sur les réseaux programmables, dont le sujet de cette étude: SDN - Réseaux Informatiques Définis par Logiciel. SDN est un nouveau paradigme conçu pour adapter les infrastructures courantes aux enjeux de la communication actuelle : haute bande passante et nature dynamique des applications.
SDN propose une nouvelle architecture plus dynamique, facile à gérer, rentable et flexible. Cette architecture découpe le plan de contrôle (intelligence et état du réseau) du plan de données (fonctions de transmission). L'approche permet de rendre le contrôle directement programmable et l'infrastructure sous-jacente d'être abstraite aux applications réseaux et services. 



\bigskip\noindent
Mots clés : SDN, Réseaux Programmables, Plan de Contrôle, Plan de Données
\end{abstract}

\hrulefill
\selectlanguage{english}
\begin{abstract}

New technologies and concepts appear in response to new user requirements such as increasingly need for high speed broadband and minimal latency. For example Big Data for massive data processing and Cloud Computing resources management and hosting. Similar evolution is expected for the computer networks which has mobilized the research community to work on programmable networks, among then the subject of this study : SDN - Software Defined Networking. SDN is a new paradigm designed to adapt current infrastructures  to the issues of the recent communication : high bandwidth and dynamic nature of the applications. SDN proposes a new architecture plus dynamic, ease to manage, profitable and flexibile. This architecture decouples the control plane (network intelligence and state) from the data plane (transmission functions). This approach makes de the control directly programmable and the underlying abstracted to network applications and servers.

\bigskip\noindent
Keywords : SDN, Programmable Networks, Control Plane, Data Plane
\end{abstract}

\hrulefill
\end{document}
